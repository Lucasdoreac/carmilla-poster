\documentclass[final,a0paper,portrait]{beamer}

% Pacotes necessários
\usepackage[scale=1.24]{beamerposter}
\usepackage[brazilian]{babel}
\usepackage[utf8]{inputenc}
\usepackage[T1]{fontenc}
\usepackage{lmodern}
\usepackage{ragged2e}
\usepackage{amsmath,amssymb}
\usepackage{booktabs}
\usepackage{graphicx}
\usepackage{multicol}
\usepackage{xcolor}
\usepackage{setspace}
% Removido: \usepackage{enumitem}

% Definição de cores ABNT
\definecolor{azultitulo}{RGB}{0, 70, 150}
\definecolor{cinzaclaro}{RGB}{240, 240, 240}

% Configuração do modo retrato (A0)
\geometry{paperwidth=84.1cm, paperheight=118.9cm, margin=2cm}

% Define o tema
\usetheme{default}
\usecolortheme{default}

% Removendo elementos desnecessários
\setbeamertemplate{navigation symbols}{}
\setbeamertemplate{headline}{}
\setbeamertemplate{footline}{}
\setbeamertemplate{itemize items}[square]

% Configurações de fonte
\setbeamerfont{title}{size=\fontsize{80}{84}\selectfont,series=\bfseries}
\setbeamerfont{subtitle}{size=\fontsize{50}{54}\selectfont}
\setbeamerfont{author}{size=\fontsize{36}{40}\selectfont}
\setbeamerfont{institute}{size=\fontsize{30}{34}\selectfont}
\setbeamerfont{block title}{size=\fontsize{42}{46}\selectfont,series=\bfseries}
\setbeamerfont{block body}{size=\fontsize{30}{34}\selectfont}

% Estilo dos blocos
\setbeamertemplate{block begin}{
  \begin{beamercolorbox}[rounded=true,shadow=true,leftskip=1cm,colsep*=.75ex]{block title}%
    \usebeamerfont{block title}\insertblocktitle
  \end{beamercolorbox}%
  \begin{beamercolorbox}[rounded=true,shadow=true,colsep*=.75ex,sep=1ex,vmode]{block body}%
    \usebeamerfont{block body}\justifying
}
\setbeamertemplate{block end}{
  \end{beamercolorbox}
}

% Cores dos blocos
\setbeamercolor{block title}{bg=azultitulo,fg=white}
\setbeamercolor{block body}{bg=cinzaclaro,fg=black}

\title{\textcolor{azultitulo}{DINÂMICAS DE PODER E RELACIONAMENTOS FEMININOS \\ EM CARMILLA:\\ UMA ANÁLISE DA CRÍTICA FEMINISTA}}

\author{Jessika Carollyne B Lessa \quad $\bullet$ \quad Lucas Dórea Cardoso \quad $\bullet$ \quad Letícia Alves Guerra \quad $\bullet$ \\
\vspace{0.3cm}
Yassanã do Nascimento Silva \quad $\bullet$ \quad Maria Alice Santos da Silva}

\institute{Literatura Inglesa e Norte-Americana (Noturno) \\ Profa. Dra. Dayse Rayane e Silva Muniz \\ Centro Universitário do Distrito Federal (UDF)}

\date{Maio de 2025}

\begin{document}
\begin{frame}[t]
\begin{columns}[t]
\begin{column}{0.98\textwidth}
\centering
\vspace{1cm}
{\usebeamerfont{title}\usebeamercolor[fg]{title}\inserttitle\par}
\vspace{0.8cm}
{\usebeamerfont{author}\usebeamercolor[fg]{author}\insertauthor\par}
\vspace{0.8cm}
{\usebeamerfont{institute}\usebeamercolor[fg]{institute}\insertinstitute\par}
\vspace{0.5cm}
\end{column}
\end{columns}

\vspace{1cm}

\begin{columns}[t]
\begin{column}{0.31\textwidth}
\begin{block}{\centering RESUMO}
Esta pesquisa analisa as dinâmicas de poder e os relacionamentos femininos na novela gótica ``Carmilla'' (1872) de Joseph Sheridan Le Fanu, sob a perspectiva da crítica feminista e queer. A obra, que antecede ``Drácula'' de Bram Stoker, apresenta uma relação complexa entre Laura, a protagonista, e Carmilla, uma vampira que desafia as normas sociais e de gênero da era vitoriana. Investigamos como Le Fanu utiliza elementos góticos para explorar temas de transgressão, autonomia feminina e desejo proibido, enquanto examina as tensões entre intimidade e perigo nos relacionamentos homossociais femininos.

\vspace{0.5cm}
\textbf{Palavras-chave:} Literatura Gótica; Crítica Feminista; Homossocialidade Feminina; Transgressão; Identidade.
\end{block}

\begin{block}{\centering INTRODUÇÃO}
A novela ``Carmilla'', publicada em 1872 por Joseph Sheridan Le Fanu, representa uma obra seminal para a compreensão das dinâmicas de poder e relacionamentos femininos na literatura vitoriana. Neste estudo, analisamos como a obra de Le Fanu desafia as estruturas patriarcais através da representação de relacionamentos íntimos entre mulheres, em um período marcado por rígidas normas sociais de gênero.

\vspace{0.5cm}
\textbf{Problema de pesquisa:} Como a relação entre Laura e Carmilla desafia e subverte os padrões de relacionamentos femininos impostos pela sociedade vitoriana, e quais são as implicações dessa transgressão para uma interpretação feminista contemporânea da obra?
\end{block}

\begin{block}{\centering OBJETIVOS}
\begin{itemize}
\item Analisar as representações de poder nas relações femininas em ``Carmilla'';
\item Identificar elementos de transgressão e subversão das normas patriarcais;
\item Examinar como o contexto vitoriano influencia a caracterização dos relacionamentos homossociais;
\item Explorar como a obra antecipa questões relevantes para a crítica feminista e queer contemporânea.
\end{itemize}
\end{block}
\end{column}

\begin{column}{0.33\textwidth}
\begin{block}{\centering CONTEXTO HISTÓRICO E SOCIAL}
\textbf{Sociedade Vitoriana e Normas de Gênero}

A sociedade vitoriana se caracterizava por uma ordem patriarcal que buscava controlar a sexualidade e autonomia femininas. As mulheres eram confinadas à esfera doméstica e esperava-se que fossem submissas e emocionalmente sensíveis, em contraste com as figuras masculinas retratadas como inteligentes e estáveis. ``Carmilla'' emerge neste contexto como uma obra que desafia essas expectativas sociais, apresentando personagens femininas que transgridem os limites estabelecidos.

\vspace{0.5cm}
\textbf{Elementos Góticos e Fronteiras Sociais}

Fred Botting observa que narrativas góticas frequentemente exploram ``transgressão e ansiedade sobre limites culturais'', o que se aplica perfeitamente à obra analisada. A ambientação isolada do castelo cercado por florestas densas cria um ambiente que espelha o isolamento social imposto às mulheres da época. Estes elementos góticos funcionam como ferramentas literárias para explorar tensões sociais.
\end{block}

\begin{block}{\centering ANÁLISE DAS DINÂMICAS DE PODER}
A relação entre Laura e Carmilla é marcada por uma complexa dinâmica de poder que oscila entre desejo e repulsa, atração e medo. Esta dualidade é representada na própria fala de Carmilla: ``Você vai me achar cruel, muito egoísta, mas o amor é sempre egoísta; quanto mais ardente, mais egoísta''. Esta autoconsciência da personagem complica a compreensão do leitor sobre moralidade e desejo no contexto dos relacionamentos femininos.

\vspace{0.5cm}
A capacidade de Carmilla de transformar sua identidade através de seus nomes anagramáticos (Millarca e Mircalla) simboliza tanto sua sexualidade transgressora quanto o potencial de as mulheres redefinirem suas identidades fora das limitações patriarcais. Esta fluidez identitária representa uma ameaça às estruturas sociais estabelecidas.
\end{block}
\end{column}

\begin{column}{0.31\textwidth}
\begin{block}{\centering RELACIONAMENTOS FEMININOS}
% Modificado: Alterado para usar o formato padrão do Beamer sem enumitem
\begin{enumerate}
\item \textbf{Subversão das Normas Patriarcais:} As interações de Carmilla com jovens mulheres evidenciam um potencial subversivo dentro dos relacionamentos femininos, desafiando as expectativas tradicionais.

\item \textbf{Homossocialidade Feminina:} Em ``Carmilla'', os vínculos homossociais femininos são retratados como simultaneamente íntimos e perigosos, refletindo a natureza dual do desejo e da vulnerabilidade.

\item \textbf{Perspectivas Queer:} A novela é rica em temas queer, que complicam a representação do amor e do desejo. Amy Leal aponta que os ``desejos inomináveis'' expressos por Carmilla podem ser interpretados como emblemáticos da experiência queer velada do período vitoriano.
\end{enumerate}
\end{block}

\begin{block}{\centering CONSIDERAÇÕES FINAIS}
Esperamos que esta pesquisa contribua para uma compreensão mais profunda de como ``Carmilla'' articula uma crítica às normas de gênero e sexualidade da era vitoriana, oferecendo insights valiosos sobre a representação de relacionamentos femininos na literatura gótica.

\vspace{0.5cm}
Antecipamos também demonstrar como os elementos góticos servem como ferramentas literárias para explorar ansiedades sociais relacionadas à autonomia feminina e ao desejo não-normativo, expandindo nossa compreensão do gênero gótico como um espaço para crítica social.
\end{block}

\begin{block}{\centering REFERÊNCIAS}
\begin{flushleft}
\footnotesize
\setstretch{1.2}

BOTTING, Fred. \textbf{Gothic}. 2. ed. London: Routledge, 2014.\\

CREED, Barbara. \textbf{The Monstrous-Feminine: Film, Feminism, Psychoanalysis}. London: Routledge, 1993.\\

GILBERT, Sandra M.; GUBAR, Susan. \textbf{The Madwoman in the Attic}. 2. ed. New Haven: Yale University Press, 2000.\\

LE FANU, Joseph Sheridan. \textbf{Carmilla}. London: Create Space, 2014. [1872]\\

LEAL, Amy. Carmilla and Sapphic Erasure. \textbf{American Gothic: An Edinburgh Companion}, 2017.\\

SIGNOROTTI, Elizabeth. Repossessing the Body: Transgressive Desire in 'Carmilla' and 'Dracula'. \textbf{Criticism}, v. 38, n. 4, p. 607-632, 1996.
\end{flushleft}
\end{block}

\begin{center}
\vspace{0.5cm}
\textcolor{azultitulo}{\large Centro Universitário do Distrito Federal (UDF) - Maio de 2025}
\end{center}

\end{column}
\end{columns}
\end{frame}
\end{document}
